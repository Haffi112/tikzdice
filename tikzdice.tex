\usepackage{etoolbox}
\usepackage{ifthen}

% Put this file into the preamble and call it via 
% \usepackage{etoolbox}
\usepackage{ifthen}

% Put this file into the preamble and call it via 
% \usepackage{etoolbox}
\usepackage{ifthen}

% Put this file into the preamble and call it via 
% \usepackage{etoolbox}
\usepackage{ifthen}

% Put this file into the preamble and call it via 
% \input{tikzdice.tex}
%
% Jesse Hamner
% https://github.com/jessehamner/tikzdice
% 2020

\newcommand{\pipsize}{3.05pt}
\newcommand{\diecolor}{red}
\newcommand{\pipcolor}{red}
\newcommand{\fieldfill}{white}

\newcommand{\maindie}[2]{
\ifthenelse{#2 = 0}{\renewcommand{\diecolor}{black}\renewcommand{\pipcolor}{black}\renewcommand{\fieldfill}{white}}{}%
\ifthenelse{#2 = 1}{\renewcommand{\diecolor}{red}\renewcommand{\pipcolor}{red}\renewcommand{\fieldfill}{white}}{}%
\ifthenelse{#2 = 2}{\renewcommand{\diecolor}{blue}\renewcommand{\pipcolor}{blue}\renewcommand{\fieldfill}{white}}{}%
\ifthenelse{#2 = 3}{\renewcommand{\diecolor}{black}\renewcommand{\pipcolor}{white}\renewcommand{\fieldfill}{black}}{}%
\begin{tikzpicture}
  \draw[very thick, rounded corners, \diecolor, fill=\fieldfill] (0,0) rectangle (1,1);
  \ifthenelse{#1 = 1}{\filldraw[fill=\pipcolor] (0.50,0.50) circle (\pipsize);}{}%  
  \ifthenelse{#1 = 2}{\filldraw[fill=\pipcolor] (0.25,0.25) circle (\pipsize);
  					  \filldraw[fill=\pipcolor] (0.75,0.75) circle (\pipsize);}{}%
  \ifthenelse{#1 = 3}{\filldraw[fill=\pipcolor] (0.25,0.25) circle (\pipsize);
				      \filldraw[fill=\pipcolor] (0.75,0.75) circle (\pipsize);
					  \filldraw[fill=\pipcolor] (0.50,0.50) circle (\pipsize);}{}%
  \ifthenelse{#1 = 4}{\filldraw[fill=\pipcolor] (0.25,0.25) circle (\pipsize);
					  \filldraw[fill=\pipcolor] (0.25,0.75) circle (\pipsize);
					  \filldraw[fill=\pipcolor] (0.75,0.25) circle (\pipsize);
					  \filldraw[fill=\pipcolor] (0.75,0.75) circle (\pipsize);}{}%
  \ifthenelse{#1 = 5}{\filldraw[fill=\pipcolor] (0.25,0.25) circle (\pipsize);
					  \filldraw[fill=\pipcolor] (0.25,0.75) circle (\pipsize);
					  \filldraw[fill=\pipcolor] (0.75,0.25) circle (\pipsize);
					  \filldraw[fill=\pipcolor] (0.75,0.75) circle (\pipsize);
					  \filldraw[fill=\pipcolor] (0.50,0.50) circle (\pipsize);}{}%
  \ifthenelse{#1 = 6}{\filldraw[fill=\pipcolor] (0.20,0.25) circle (\pipsize);
					  \filldraw[fill=\pipcolor] (0.20,0.75) circle (\pipsize);
					  \filldraw[fill=\pipcolor] (0.80,0.25) circle (\pipsize);
					  \filldraw[fill=\pipcolor] (0.80,0.75) circle (\pipsize);
					  \filldraw[fill=\pipcolor] (0.50,0.75) circle (\pipsize);
					  \filldraw[fill=\pipcolor] (0.50,0.25) circle (\pipsize);}{}%
	\ifthenelse{#1 = 7}{\filldraw[fill=\pipcolor] (0.20,0.20) circle (\pipsize);
					  \filldraw[fill=\pipcolor] (0.20,0.80) circle (\pipsize);
					  \filldraw[fill=\pipcolor] (0.80,0.20) circle (\pipsize);
					  \filldraw[fill=\pipcolor] (0.80,0.80) circle (\pipsize);
					  \filldraw[fill=\pipcolor] (0.50,0.80) circle (\pipsize);
  					  \filldraw[fill=\pipcolor] (0.50,0.50) circle (\pipsize);
					  \filldraw[fill=\pipcolor] (0.50,0.20) circle (\pipsize);}{}%	  
	\ifthenelse{#1 = 8}{\filldraw[fill=\pipcolor] (0.20,0.20) circle (\pipsize);
					  \filldraw[fill=\pipcolor] (0.20,0.50) circle (\pipsize);
					  \filldraw[fill=\pipcolor] (0.20,0.80) circle (\pipsize);
					  \filldraw[fill=\pipcolor] (0.80,0.20) circle (\pipsize);
					  \filldraw[fill=\pipcolor] (0.80,0.50) circle (\pipsize);
					  \filldraw[fill=\pipcolor] (0.80,0.80) circle (\pipsize);
					  \filldraw[fill=\pipcolor] (0.50,0.80) circle (\pipsize);
					  \filldraw[fill=\pipcolor] (0.50,0.20) circle (\pipsize);}{}%	  
    \ifthenelse{#1 = 9}{\filldraw[fill=\pipcolor] (0.20,0.20) circle (\pipsize);
					  \filldraw[fill=\pipcolor] (0.20,0.50) circle (\pipsize);
					  \filldraw[fill=\pipcolor] (0.20,0.80) circle (\pipsize);
 					  \filldraw[fill=\pipcolor] (0.20,0.50) circle (\pipsize);
					  \filldraw[fill=\pipcolor] (0.80,0.20) circle (\pipsize);
					  \filldraw[fill=\pipcolor] (0.80,0.50) circle (\pipsize);
					  \filldraw[fill=\pipcolor] (0.80,0.80) circle (\pipsize);
					  \filldraw[fill=\pipcolor] (0.50,0.80) circle (\pipsize);
  					  \filldraw[fill=\pipcolor] (0.50,0.50) circle (\pipsize);
					  \filldraw[fill=\pipcolor] (0.50,0.20) circle (\pipsize);}{}%	  
					  
\end{tikzpicture}
}

\newcommand{\dieone}{\maindie{1}{0}}
\newcommand{\dietwo}{\maindie{2}{0}}
\newcommand{\diethree}{\maindie{3}{0}}
\newcommand{\diefour}{\maindie{4}{0}}
\newcommand{\diefive}{\maindie{5}{0}}
\newcommand{\diesix}{\maindie{6}{0}}

%
% Jesse Hamner
% https://github.com/jessehamner/tikzdice
% 2020

\newcommand{\pipsize}{3.05pt}
\newcommand{\diecolor}{red}
\newcommand{\pipcolor}{red}
\newcommand{\fieldfill}{white}

\newcommand{\maindie}[2]{
\ifthenelse{#2 = 0}{\renewcommand{\diecolor}{black}\renewcommand{\pipcolor}{black}\renewcommand{\fieldfill}{white}}{}%
\ifthenelse{#2 = 1}{\renewcommand{\diecolor}{red}\renewcommand{\pipcolor}{red}\renewcommand{\fieldfill}{white}}{}%
\ifthenelse{#2 = 2}{\renewcommand{\diecolor}{blue}\renewcommand{\pipcolor}{blue}\renewcommand{\fieldfill}{white}}{}%
\ifthenelse{#2 = 3}{\renewcommand{\diecolor}{black}\renewcommand{\pipcolor}{white}\renewcommand{\fieldfill}{black}}{}%
\begin{tikzpicture}
  \draw[very thick, rounded corners, \diecolor, fill=\fieldfill] (0,0) rectangle (1,1);
  \ifthenelse{#1 = 1}{\filldraw[fill=\pipcolor] (0.50,0.50) circle (\pipsize);}{}%  
  \ifthenelse{#1 = 2}{\filldraw[fill=\pipcolor] (0.25,0.25) circle (\pipsize);
  					  \filldraw[fill=\pipcolor] (0.75,0.75) circle (\pipsize);}{}%
  \ifthenelse{#1 = 3}{\filldraw[fill=\pipcolor] (0.25,0.25) circle (\pipsize);
				      \filldraw[fill=\pipcolor] (0.75,0.75) circle (\pipsize);
					  \filldraw[fill=\pipcolor] (0.50,0.50) circle (\pipsize);}{}%
  \ifthenelse{#1 = 4}{\filldraw[fill=\pipcolor] (0.25,0.25) circle (\pipsize);
					  \filldraw[fill=\pipcolor] (0.25,0.75) circle (\pipsize);
					  \filldraw[fill=\pipcolor] (0.75,0.25) circle (\pipsize);
					  \filldraw[fill=\pipcolor] (0.75,0.75) circle (\pipsize);}{}%
  \ifthenelse{#1 = 5}{\filldraw[fill=\pipcolor] (0.25,0.25) circle (\pipsize);
					  \filldraw[fill=\pipcolor] (0.25,0.75) circle (\pipsize);
					  \filldraw[fill=\pipcolor] (0.75,0.25) circle (\pipsize);
					  \filldraw[fill=\pipcolor] (0.75,0.75) circle (\pipsize);
					  \filldraw[fill=\pipcolor] (0.50,0.50) circle (\pipsize);}{}%
  \ifthenelse{#1 = 6}{\filldraw[fill=\pipcolor] (0.20,0.25) circle (\pipsize);
					  \filldraw[fill=\pipcolor] (0.20,0.75) circle (\pipsize);
					  \filldraw[fill=\pipcolor] (0.80,0.25) circle (\pipsize);
					  \filldraw[fill=\pipcolor] (0.80,0.75) circle (\pipsize);
					  \filldraw[fill=\pipcolor] (0.50,0.75) circle (\pipsize);
					  \filldraw[fill=\pipcolor] (0.50,0.25) circle (\pipsize);}{}%
	\ifthenelse{#1 = 7}{\filldraw[fill=\pipcolor] (0.20,0.20) circle (\pipsize);
					  \filldraw[fill=\pipcolor] (0.20,0.80) circle (\pipsize);
					  \filldraw[fill=\pipcolor] (0.80,0.20) circle (\pipsize);
					  \filldraw[fill=\pipcolor] (0.80,0.80) circle (\pipsize);
					  \filldraw[fill=\pipcolor] (0.50,0.80) circle (\pipsize);
  					  \filldraw[fill=\pipcolor] (0.50,0.50) circle (\pipsize);
					  \filldraw[fill=\pipcolor] (0.50,0.20) circle (\pipsize);}{}%	  
	\ifthenelse{#1 = 8}{\filldraw[fill=\pipcolor] (0.20,0.20) circle (\pipsize);
					  \filldraw[fill=\pipcolor] (0.20,0.50) circle (\pipsize);
					  \filldraw[fill=\pipcolor] (0.20,0.80) circle (\pipsize);
					  \filldraw[fill=\pipcolor] (0.80,0.20) circle (\pipsize);
					  \filldraw[fill=\pipcolor] (0.80,0.50) circle (\pipsize);
					  \filldraw[fill=\pipcolor] (0.80,0.80) circle (\pipsize);
					  \filldraw[fill=\pipcolor] (0.50,0.80) circle (\pipsize);
					  \filldraw[fill=\pipcolor] (0.50,0.20) circle (\pipsize);}{}%	  
    \ifthenelse{#1 = 9}{\filldraw[fill=\pipcolor] (0.20,0.20) circle (\pipsize);
					  \filldraw[fill=\pipcolor] (0.20,0.50) circle (\pipsize);
					  \filldraw[fill=\pipcolor] (0.20,0.80) circle (\pipsize);
 					  \filldraw[fill=\pipcolor] (0.20,0.50) circle (\pipsize);
					  \filldraw[fill=\pipcolor] (0.80,0.20) circle (\pipsize);
					  \filldraw[fill=\pipcolor] (0.80,0.50) circle (\pipsize);
					  \filldraw[fill=\pipcolor] (0.80,0.80) circle (\pipsize);
					  \filldraw[fill=\pipcolor] (0.50,0.80) circle (\pipsize);
  					  \filldraw[fill=\pipcolor] (0.50,0.50) circle (\pipsize);
					  \filldraw[fill=\pipcolor] (0.50,0.20) circle (\pipsize);}{}%	  
					  
\end{tikzpicture}
}

\newcommand{\dieone}{\maindie{1}{0}}
\newcommand{\dietwo}{\maindie{2}{0}}
\newcommand{\diethree}{\maindie{3}{0}}
\newcommand{\diefour}{\maindie{4}{0}}
\newcommand{\diefive}{\maindie{5}{0}}
\newcommand{\diesix}{\maindie{6}{0}}

%
% Jesse Hamner
% https://github.com/jessehamner/tikzdice
% 2020

\newcommand{\pipsize}{3.05pt}
\newcommand{\diecolor}{red}
\newcommand{\pipcolor}{red}
\newcommand{\fieldfill}{white}

\newcommand{\maindie}[2]{
\ifthenelse{#2 = 0}{\renewcommand{\diecolor}{black}\renewcommand{\pipcolor}{black}\renewcommand{\fieldfill}{white}}{}%
\ifthenelse{#2 = 1}{\renewcommand{\diecolor}{red}\renewcommand{\pipcolor}{red}\renewcommand{\fieldfill}{white}}{}%
\ifthenelse{#2 = 2}{\renewcommand{\diecolor}{blue}\renewcommand{\pipcolor}{blue}\renewcommand{\fieldfill}{white}}{}%
\ifthenelse{#2 = 3}{\renewcommand{\diecolor}{black}\renewcommand{\pipcolor}{white}\renewcommand{\fieldfill}{black}}{}%
\begin{tikzpicture}
  \draw[very thick, rounded corners, \diecolor, fill=\fieldfill] (0,0) rectangle (1,1);
  \ifthenelse{#1 = 1}{\filldraw[fill=\pipcolor] (0.50,0.50) circle (\pipsize);}{}%  
  \ifthenelse{#1 = 2}{\filldraw[fill=\pipcolor] (0.25,0.25) circle (\pipsize);
  					  \filldraw[fill=\pipcolor] (0.75,0.75) circle (\pipsize);}{}%
  \ifthenelse{#1 = 3}{\filldraw[fill=\pipcolor] (0.25,0.25) circle (\pipsize);
				      \filldraw[fill=\pipcolor] (0.75,0.75) circle (\pipsize);
					  \filldraw[fill=\pipcolor] (0.50,0.50) circle (\pipsize);}{}%
  \ifthenelse{#1 = 4}{\filldraw[fill=\pipcolor] (0.25,0.25) circle (\pipsize);
					  \filldraw[fill=\pipcolor] (0.25,0.75) circle (\pipsize);
					  \filldraw[fill=\pipcolor] (0.75,0.25) circle (\pipsize);
					  \filldraw[fill=\pipcolor] (0.75,0.75) circle (\pipsize);}{}%
  \ifthenelse{#1 = 5}{\filldraw[fill=\pipcolor] (0.25,0.25) circle (\pipsize);
					  \filldraw[fill=\pipcolor] (0.25,0.75) circle (\pipsize);
					  \filldraw[fill=\pipcolor] (0.75,0.25) circle (\pipsize);
					  \filldraw[fill=\pipcolor] (0.75,0.75) circle (\pipsize);
					  \filldraw[fill=\pipcolor] (0.50,0.50) circle (\pipsize);}{}%
  \ifthenelse{#1 = 6}{\filldraw[fill=\pipcolor] (0.20,0.25) circle (\pipsize);
					  \filldraw[fill=\pipcolor] (0.20,0.75) circle (\pipsize);
					  \filldraw[fill=\pipcolor] (0.80,0.25) circle (\pipsize);
					  \filldraw[fill=\pipcolor] (0.80,0.75) circle (\pipsize);
					  \filldraw[fill=\pipcolor] (0.50,0.75) circle (\pipsize);
					  \filldraw[fill=\pipcolor] (0.50,0.25) circle (\pipsize);}{}%
	\ifthenelse{#1 = 7}{\filldraw[fill=\pipcolor] (0.20,0.20) circle (\pipsize);
					  \filldraw[fill=\pipcolor] (0.20,0.80) circle (\pipsize);
					  \filldraw[fill=\pipcolor] (0.80,0.20) circle (\pipsize);
					  \filldraw[fill=\pipcolor] (0.80,0.80) circle (\pipsize);
					  \filldraw[fill=\pipcolor] (0.50,0.80) circle (\pipsize);
  					  \filldraw[fill=\pipcolor] (0.50,0.50) circle (\pipsize);
					  \filldraw[fill=\pipcolor] (0.50,0.20) circle (\pipsize);}{}%	  
	\ifthenelse{#1 = 8}{\filldraw[fill=\pipcolor] (0.20,0.20) circle (\pipsize);
					  \filldraw[fill=\pipcolor] (0.20,0.50) circle (\pipsize);
					  \filldraw[fill=\pipcolor] (0.20,0.80) circle (\pipsize);
					  \filldraw[fill=\pipcolor] (0.80,0.20) circle (\pipsize);
					  \filldraw[fill=\pipcolor] (0.80,0.50) circle (\pipsize);
					  \filldraw[fill=\pipcolor] (0.80,0.80) circle (\pipsize);
					  \filldraw[fill=\pipcolor] (0.50,0.80) circle (\pipsize);
					  \filldraw[fill=\pipcolor] (0.50,0.20) circle (\pipsize);}{}%	  
    \ifthenelse{#1 = 9}{\filldraw[fill=\pipcolor] (0.20,0.20) circle (\pipsize);
					  \filldraw[fill=\pipcolor] (0.20,0.50) circle (\pipsize);
					  \filldraw[fill=\pipcolor] (0.20,0.80) circle (\pipsize);
 					  \filldraw[fill=\pipcolor] (0.20,0.50) circle (\pipsize);
					  \filldraw[fill=\pipcolor] (0.80,0.20) circle (\pipsize);
					  \filldraw[fill=\pipcolor] (0.80,0.50) circle (\pipsize);
					  \filldraw[fill=\pipcolor] (0.80,0.80) circle (\pipsize);
					  \filldraw[fill=\pipcolor] (0.50,0.80) circle (\pipsize);
  					  \filldraw[fill=\pipcolor] (0.50,0.50) circle (\pipsize);
					  \filldraw[fill=\pipcolor] (0.50,0.20) circle (\pipsize);}{}%	  
					  
\end{tikzpicture}
}

\newcommand{\dieone}{\maindie{1}{0}}
\newcommand{\dietwo}{\maindie{2}{0}}
\newcommand{\diethree}{\maindie{3}{0}}
\newcommand{\diefour}{\maindie{4}{0}}
\newcommand{\diefive}{\maindie{5}{0}}
\newcommand{\diesix}{\maindie{6}{0}}

%
% Jesse Hamner
% https://github.com/jessehamner/tikzdice
% 2020

\newcommand{\pipsize}{3.05pt}
\newcommand{\diecolor}{red}
\newcommand{\pipcolor}{red}
\newcommand{\fieldfill}{white}

\newcommand{\maindie}[2]{
\ifthenelse{#2 = 0}{\renewcommand{\diecolor}{black}\renewcommand{\pipcolor}{black}\renewcommand{\fieldfill}{white}}{}%
\ifthenelse{#2 = 1}{\renewcommand{\diecolor}{red}\renewcommand{\pipcolor}{red}\renewcommand{\fieldfill}{white}}{}%
\ifthenelse{#2 = 2}{\renewcommand{\diecolor}{blue}\renewcommand{\pipcolor}{blue}\renewcommand{\fieldfill}{white}}{}%
\ifthenelse{#2 = 3}{\renewcommand{\diecolor}{black}\renewcommand{\pipcolor}{white}\renewcommand{\fieldfill}{black}}{}%
\begin{tikzpicture}
  \draw[very thick, rounded corners, \diecolor, fill=\fieldfill] (0,0) rectangle (1,1);
  \ifthenelse{#1 = 1}{\filldraw[fill=\pipcolor] (0.50,0.50) circle (\pipsize);}{}%  
  \ifthenelse{#1 = 2}{\filldraw[fill=\pipcolor] (0.25,0.25) circle (\pipsize);
  					  \filldraw[fill=\pipcolor] (0.75,0.75) circle (\pipsize);}{}%
  \ifthenelse{#1 = 3}{\filldraw[fill=\pipcolor] (0.25,0.25) circle (\pipsize);
				      \filldraw[fill=\pipcolor] (0.75,0.75) circle (\pipsize);
					  \filldraw[fill=\pipcolor] (0.50,0.50) circle (\pipsize);}{}%
  \ifthenelse{#1 = 4}{\filldraw[fill=\pipcolor] (0.25,0.25) circle (\pipsize);
					  \filldraw[fill=\pipcolor] (0.25,0.75) circle (\pipsize);
					  \filldraw[fill=\pipcolor] (0.75,0.25) circle (\pipsize);
					  \filldraw[fill=\pipcolor] (0.75,0.75) circle (\pipsize);}{}%
  \ifthenelse{#1 = 5}{\filldraw[fill=\pipcolor] (0.25,0.25) circle (\pipsize);
					  \filldraw[fill=\pipcolor] (0.25,0.75) circle (\pipsize);
					  \filldraw[fill=\pipcolor] (0.75,0.25) circle (\pipsize);
					  \filldraw[fill=\pipcolor] (0.75,0.75) circle (\pipsize);
					  \filldraw[fill=\pipcolor] (0.50,0.50) circle (\pipsize);}{}%
  \ifthenelse{#1 = 6}{\filldraw[fill=\pipcolor] (0.20,0.25) circle (\pipsize);
					  \filldraw[fill=\pipcolor] (0.20,0.75) circle (\pipsize);
					  \filldraw[fill=\pipcolor] (0.80,0.25) circle (\pipsize);
					  \filldraw[fill=\pipcolor] (0.80,0.75) circle (\pipsize);
					  \filldraw[fill=\pipcolor] (0.50,0.75) circle (\pipsize);
					  \filldraw[fill=\pipcolor] (0.50,0.25) circle (\pipsize);}{}%
	\ifthenelse{#1 = 7}{\filldraw[fill=\pipcolor] (0.20,0.20) circle (\pipsize);
					  \filldraw[fill=\pipcolor] (0.20,0.80) circle (\pipsize);
					  \filldraw[fill=\pipcolor] (0.80,0.20) circle (\pipsize);
					  \filldraw[fill=\pipcolor] (0.80,0.80) circle (\pipsize);
					  \filldraw[fill=\pipcolor] (0.50,0.80) circle (\pipsize);
  					  \filldraw[fill=\pipcolor] (0.50,0.50) circle (\pipsize);
					  \filldraw[fill=\pipcolor] (0.50,0.20) circle (\pipsize);}{}%	  
	\ifthenelse{#1 = 8}{\filldraw[fill=\pipcolor] (0.20,0.20) circle (\pipsize);
					  \filldraw[fill=\pipcolor] (0.20,0.50) circle (\pipsize);
					  \filldraw[fill=\pipcolor] (0.20,0.80) circle (\pipsize);
					  \filldraw[fill=\pipcolor] (0.80,0.20) circle (\pipsize);
					  \filldraw[fill=\pipcolor] (0.80,0.50) circle (\pipsize);
					  \filldraw[fill=\pipcolor] (0.80,0.80) circle (\pipsize);
					  \filldraw[fill=\pipcolor] (0.50,0.80) circle (\pipsize);
					  \filldraw[fill=\pipcolor] (0.50,0.20) circle (\pipsize);}{}%	  
    \ifthenelse{#1 = 9}{\filldraw[fill=\pipcolor] (0.20,0.20) circle (\pipsize);
					  \filldraw[fill=\pipcolor] (0.20,0.50) circle (\pipsize);
					  \filldraw[fill=\pipcolor] (0.20,0.80) circle (\pipsize);
 					  \filldraw[fill=\pipcolor] (0.20,0.50) circle (\pipsize);
					  \filldraw[fill=\pipcolor] (0.80,0.20) circle (\pipsize);
					  \filldraw[fill=\pipcolor] (0.80,0.50) circle (\pipsize);
					  \filldraw[fill=\pipcolor] (0.80,0.80) circle (\pipsize);
					  \filldraw[fill=\pipcolor] (0.50,0.80) circle (\pipsize);
  					  \filldraw[fill=\pipcolor] (0.50,0.50) circle (\pipsize);
					  \filldraw[fill=\pipcolor] (0.50,0.20) circle (\pipsize);}{}%	  
					  
\end{tikzpicture}
}

\newcommand{\dieone}{\maindie{1}{0}}
\newcommand{\dietwo}{\maindie{2}{0}}
\newcommand{\diethree}{\maindie{3}{0}}
\newcommand{\diefour}{\maindie{4}{0}}
\newcommand{\diefive}{\maindie{5}{0}}
\newcommand{\diesix}{\maindie{6}{0}}
